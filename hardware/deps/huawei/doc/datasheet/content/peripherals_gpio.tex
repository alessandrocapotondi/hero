\section{GPIO}

\begin{table}[H]
 \caption{GPIO Signals}
 \label{tab:gpio_signals}
  \begin{tabularx}{\textwidth}{@{}llX@{}} \toprule
    \textbf{Signal}                  & \textbf{Direction} & \textbf{Description}         \\ \toprule
    \signal{gpio\_in[31:0]}          & \textbf{input}     & Transmit Data                \\ \hline
    \signal{gpio\_out[31:0]}         & \textbf{output}    & Receive Data                 \\ \hline
    \signal{gpio\_dir[31:0]}         & \textbf{output}    & Request to Send              \\ \hline
    \signal{gpio\_padcfg[5:0][31:0]} & \textbf{output}    & Pad Configuration            \\ \hline
    \signal{interrupt}               & \textbf{output}    & Interrupt (Rise or Fall or Level)\\ \hline
  \end{tabularx}
\end{table}

\regDesc{0x1A10\_1000}{0x0000\_0000}{PADDIR (Pad Direction)}{
  \begin{bytefield}[rightcurly=.,endianness=big]{32}
  \bitheader{31,30,29,28,27,26,25,24,23,22,21,20,19,18,17,16,15,14,13,12,11,10,9,8,7,6,5,4,3,2,1,0} \\
  \begin{rightwordgroup}{PADDIR}
    \bitbox{1}{\tiny D}
    \bitbox{1}{\tiny D}
    \bitbox{1}{\tiny D}
    \bitbox{1}{\tiny D}
    \bitbox{1}{\tiny D}
    \bitbox{1}{\tiny D}
    \bitbox{1}{\tiny D}
    \bitbox{1}{\tiny D}
    \bitbox{1}{\tiny D}
    \bitbox{1}{\tiny D}
    \bitbox{1}{\tiny D}
    \bitbox{1}{\tiny D}
    \bitbox{1}{\tiny D}
    \bitbox{1}{\tiny D}
    \bitbox{1}{\tiny D}
    \bitbox{1}{\tiny D}
    \bitbox{1}{\tiny D}
    \bitbox{1}{\tiny D}
    \bitbox{1}{\tiny D}
    \bitbox{1}{\tiny D}
    \bitbox{1}{\tiny D}
    \bitbox{1}{\tiny D}
    \bitbox{1}{\tiny D}
    \bitbox{1}{\tiny D}
    \bitbox{1}{\tiny D}
    \bitbox{1}{\tiny D}
    \bitbox{1}{\tiny D}
    \bitbox{1}{\tiny D}
    \bitbox{1}{\tiny D}
    \bitbox{1}{\tiny D}
    \bitbox{1}{\tiny D}
    \bitbox{1}{\tiny D}
  \end{rightwordgroup}\\
  \end{bytefield}
}{
  \regItem{Bit 31:0}{PADDIR}{Pad Direction.\\
    Control the direction of each of the GPIO pads. A value of \signal{1} means
    it is configured as an output, while \signal{0} configures it as an input.
  }
}

\regDesc{0x1A10\_1004}{0x0000\_0000}{PADIN (Input Values)}{
  \begin{bytefield}[rightcurly=.,endianness=big]{32}
  \bitheader{31,30,29,28,27,26,25,24,23,22,21,20,19,18,17,16,15,14,13,12,11,10,9,8,7,6,5,4,3,2,1,0} \\
  \begin{rightwordgroup}{PADIN}
    \bitbox{1}{\tiny I}
    \bitbox{1}{\tiny I}
    \bitbox{1}{\tiny I}
    \bitbox{1}{\tiny I}
    \bitbox{1}{\tiny I}
    \bitbox{1}{\tiny I}
    \bitbox{1}{\tiny I}
    \bitbox{1}{\tiny I}
    \bitbox{1}{\tiny I}
    \bitbox{1}{\tiny I}
    \bitbox{1}{\tiny I}
    \bitbox{1}{\tiny I}
    \bitbox{1}{\tiny I}
    \bitbox{1}{\tiny I}
    \bitbox{1}{\tiny I}
    \bitbox{1}{\tiny I}
    \bitbox{1}{\tiny I}
    \bitbox{1}{\tiny I}
    \bitbox{1}{\tiny I}
    \bitbox{1}{\tiny I}
    \bitbox{1}{\tiny I}
    \bitbox{1}{\tiny I}
    \bitbox{1}{\tiny I}
    \bitbox{1}{\tiny I}
    \bitbox{1}{\tiny I}
    \bitbox{1}{\tiny I}
    \bitbox{1}{\tiny I}
    \bitbox{1}{\tiny I}
    \bitbox{1}{\tiny I}
    \bitbox{1}{\tiny I}
    \bitbox{1}{\tiny I}
    \bitbox{1}{\tiny I}
  \end{rightwordgroup}\\
  \end{bytefield}
}{
  \regItem{Bit 31:0}{PADIN}{Input Values.
  }
}

\regDesc{0x1A10\_1008}{0x0000\_0000}{PADOUT (Output Values)}{
  \begin{bytefield}[rightcurly=.,endianness=big]{32}
  \bitheader{31,30,29,28,27,26,25,24,23,22,21,20,19,18,17,16,15,14,13,12,11,10,9,8,7,6,5,4,3,2,1,0} \\
  \begin{rightwordgroup}{PADOUT}
    \bitbox{1}{\tiny O}
    \bitbox{1}{\tiny O}
    \bitbox{1}{\tiny O}
    \bitbox{1}{\tiny O}
    \bitbox{1}{\tiny O}
    \bitbox{1}{\tiny O}
    \bitbox{1}{\tiny O}
    \bitbox{1}{\tiny O}
    \bitbox{1}{\tiny O}
    \bitbox{1}{\tiny O}
    \bitbox{1}{\tiny O}
    \bitbox{1}{\tiny O}
    \bitbox{1}{\tiny O}
    \bitbox{1}{\tiny O}
    \bitbox{1}{\tiny O}
    \bitbox{1}{\tiny O}
    \bitbox{1}{\tiny O}
    \bitbox{1}{\tiny O}
    \bitbox{1}{\tiny O}
    \bitbox{1}{\tiny O}
    \bitbox{1}{\tiny O}
    \bitbox{1}{\tiny O}
    \bitbox{1}{\tiny O}
    \bitbox{1}{\tiny O}
    \bitbox{1}{\tiny O}
    \bitbox{1}{\tiny O}
    \bitbox{1}{\tiny O}
    \bitbox{1}{\tiny O}
    \bitbox{1}{\tiny O}
    \bitbox{1}{\tiny O}
    \bitbox{1}{\tiny O}
    \bitbox{1}{\tiny O}
  \end{rightwordgroup}\\
  \end{bytefield}
}{
  \regItem{Bit 31:0}{PADOUT}{Output Values.
  }
}

\regDesc{0x1A10\_100C}{0x0000\_0000}{INTEN (Interrupt Enable)}{
  \begin{bytefield}[rightcurly=.,endianness=big]{32}
  \bitheader{31,30,29,28,27,26,25,24,23,22,21,20,19,18,17,16,15,14,13,12,11,10,9,8,7,6,5,4,3,2,1,0} \\
  \begin{rightwordgroup}{INTEN}
    \bitbox{1}{\tiny IT}
    \bitbox{1}{\tiny IT}
    \bitbox{1}{\tiny IT}
    \bitbox{1}{\tiny IT}
    \bitbox{1}{\tiny IT}
    \bitbox{1}{\tiny IT}
    \bitbox{1}{\tiny IT}
    \bitbox{1}{\tiny IT}
    \bitbox{1}{\tiny IT}
    \bitbox{1}{\tiny IT}
    \bitbox{1}{\tiny IT}
    \bitbox{1}{\tiny IT}
    \bitbox{1}{\tiny IT}
    \bitbox{1}{\tiny IT}
    \bitbox{1}{\tiny IT}
    \bitbox{1}{\tiny IT}
    \bitbox{1}{\tiny IT}
    \bitbox{1}{\tiny IT}
    \bitbox{1}{\tiny IT}
    \bitbox{1}{\tiny IT}
    \bitbox{1}{\tiny IT}
    \bitbox{1}{\tiny IT}
    \bitbox{1}{\tiny IT}
    \bitbox{1}{\tiny IT}
    \bitbox{1}{\tiny IT}
    \bitbox{1}{\tiny IT}
    \bitbox{1}{\tiny IT}
    \bitbox{1}{\tiny IT}
    \bitbox{1}{\tiny IT}
    \bitbox{1}{\tiny IT}
    \bitbox{1}{\tiny IT}
    \bitbox{1}{\tiny IT}
  \end{rightwordgroup}\\
  \end{bytefield}
}{
  \regItem{Bit 31:0}{INTEN}{Interrupt Enable. \\
    Interrupt enable per input bit. INTTYPE0 and INTTYPE1 control the interrupt
    triggering behavior.

    There are four triggers available
    \begin{itemize}
      \item \signal{INTTYPE0 = 0, INTTYPE1 = 0}: Level 1
      \item \signal{INTTYPE0 = 1, INTTYPE1 = 0}: Level 0
      \item \signal{INTTYPE0 = 0, INTTYPE1 = 1}: Rise
      \item \signal{INTTYPE0 = 1, INTTYPE1 = 1}: Fall
    \end{itemize}
  }
}

\regDesc{0x1A10\_1010}{0x0000\_0000}{INTTYPE0 (Interrupt Type 0)}{
  \begin{bytefield}[rightcurly=.,endianness=big]{32}
  \bitheader{31,30,29,28,27,26,25,24,23,22,21,20,19,18,17,16,15,14,13,12,11,10,9,8,7,6,5,4,3,2,1,0} \\
  \begin{rightwordgroup}{INTTYPE0}
    \bitbox{1}{\tiny T0}
    \bitbox{1}{\tiny T0}
    \bitbox{1}{\tiny T0}
    \bitbox{1}{\tiny T0}
    \bitbox{1}{\tiny T0}
    \bitbox{1}{\tiny T0}
    \bitbox{1}{\tiny T0}
    \bitbox{1}{\tiny T0}
    \bitbox{1}{\tiny T0}
    \bitbox{1}{\tiny T0}
    \bitbox{1}{\tiny T0}
    \bitbox{1}{\tiny T0}
    \bitbox{1}{\tiny T0}
    \bitbox{1}{\tiny T0}
    \bitbox{1}{\tiny T0}
    \bitbox{1}{\tiny T0}
    \bitbox{1}{\tiny T0}
    \bitbox{1}{\tiny T0}
    \bitbox{1}{\tiny T0}
    \bitbox{1}{\tiny T0}
    \bitbox{1}{\tiny T0}
    \bitbox{1}{\tiny T0}
    \bitbox{1}{\tiny T0}
    \bitbox{1}{\tiny T0}
    \bitbox{1}{\tiny T0}
    \bitbox{1}{\tiny T0}
    \bitbox{1}{\tiny T0}
    \bitbox{1}{\tiny T0}
    \bitbox{1}{\tiny T0}
    \bitbox{1}{\tiny T0}
    \bitbox{1}{\tiny T0}
    \bitbox{1}{\tiny T0}
  \end{rightwordgroup}\\
  \end{bytefield}
}{
  \regItem{Bit 31:0}{INTTYPE0}{Interrupt Type 0. \\
    Controls the interrupt trigger behavior together with INTTYPE1. Use INTEN to
    enable interrupts first.
  }
}

\regDesc{0x1A10\_1014}{0x0000\_0000}{INTTYPE1 (Interrupt Type 1)}{
  \begin{bytefield}[rightcurly=.,endianness=big]{32}
  \bitheader{31,30,29,28,27,26,25,24,23,22,21,20,19,18,17,16,15,14,13,12,11,10,9,8,7,6,5,4,3,2,1,0} \\
  \begin{rightwordgroup}{INTTYPE1}
    \bitbox{1}{\tiny T1}
    \bitbox{1}{\tiny T1}
    \bitbox{1}{\tiny T1}
    \bitbox{1}{\tiny T1}
    \bitbox{1}{\tiny T1}
    \bitbox{1}{\tiny T1}
    \bitbox{1}{\tiny T1}
    \bitbox{1}{\tiny T1}
    \bitbox{1}{\tiny T1}
    \bitbox{1}{\tiny T1}
    \bitbox{1}{\tiny T1}
    \bitbox{1}{\tiny T1}
    \bitbox{1}{\tiny T1}
    \bitbox{1}{\tiny T1}
    \bitbox{1}{\tiny T1}
    \bitbox{1}{\tiny T1}
    \bitbox{1}{\tiny T1}
    \bitbox{1}{\tiny T1}
    \bitbox{1}{\tiny T1}
    \bitbox{1}{\tiny T1}
    \bitbox{1}{\tiny T1}
    \bitbox{1}{\tiny T1}
    \bitbox{1}{\tiny T1}
    \bitbox{1}{\tiny T1}
    \bitbox{1}{\tiny T1}
    \bitbox{1}{\tiny T1}
    \bitbox{1}{\tiny T1}
    \bitbox{1}{\tiny T1}
    \bitbox{1}{\tiny T1}
    \bitbox{1}{\tiny T1}
    \bitbox{1}{\tiny T1}
    \bitbox{1}{\tiny T1}
  \end{rightwordgroup}\\
  \end{bytefield}
}{
  \regItem{Bit 31:0}{INTTYPE1}{Interrupt Type 1. \\
    Controls the interrupt trigger behavior together with INTTYPE0. Use INTEN to
    enable interrupts first.
  }
}

\regDesc{0x1A10\_1018}{0x0000\_0000}{INTSTATUS (Interrupt Status)}{
  \begin{bytefield}[rightcurly=.,endianness=big]{32}
  \bitheader{31,30,29,28,27,26,25,24,23,22,21,20,19,18,17,16,15,14,13,12,11,10,9,8,7,6,5,4,3,2,1,0} \\
  \begin{rightwordgroup}{INTSTATUS}
    \bitbox{1}{\tiny S}
    \bitbox{1}{\tiny S}
    \bitbox{1}{\tiny S}
    \bitbox{1}{\tiny S}
    \bitbox{1}{\tiny S}
    \bitbox{1}{\tiny S}
    \bitbox{1}{\tiny S}
    \bitbox{1}{\tiny S}
    \bitbox{1}{\tiny S}
    \bitbox{1}{\tiny S}
    \bitbox{1}{\tiny S}
    \bitbox{1}{\tiny S}
    \bitbox{1}{\tiny S}
    \bitbox{1}{\tiny S}
    \bitbox{1}{\tiny S}
    \bitbox{1}{\tiny S}
    \bitbox{1}{\tiny S}
    \bitbox{1}{\tiny S}
    \bitbox{1}{\tiny S}
    \bitbox{1}{\tiny S}
    \bitbox{1}{\tiny S}
    \bitbox{1}{\tiny S}
    \bitbox{1}{\tiny S}
    \bitbox{1}{\tiny S}
    \bitbox{1}{\tiny S}
    \bitbox{1}{\tiny S}
    \bitbox{1}{\tiny S}
    \bitbox{1}{\tiny S}
    \bitbox{1}{\tiny S}
    \bitbox{1}{\tiny S}
    \bitbox{1}{\tiny S}
    \bitbox{1}{\tiny S}
  \end{rightwordgroup}\\
  \end{bytefield}
}{
  \regItem{Bit 31:0}{INTSTATUS}{Interrupt Status. \\
    Contains interrupt status per GPIO line. The status register is cleared when
    read. Similarly the \signal{interrupt} line is high while a bit is set in
    interrupt status and will be deasserted when the status register is read.
  }
}

\regDesc{0x1A10\_101C}{0x0000\_0000}{GPIOEN (GPIO Enable)}{
  \begin{bytefield}[rightcurly=.,endianness=big]{32}
  \bitheader{31,30,29,28,27,26,25,24,23,22,21,20,19,18,17,16,15,14,13,12,11,10,9,8,7,6,5,4,3,2,1,0} \\
  \begin{rightwordgroup}{GPIOEN}
    \bitbox{1}{\tiny S}
    \bitbox{1}{\tiny S}
    \bitbox{1}{\tiny S}
    \bitbox{1}{\tiny S}
    \bitbox{1}{\tiny S}
    \bitbox{1}{\tiny S}
    \bitbox{1}{\tiny S}
    \bitbox{1}{\tiny S}
    \bitbox{1}{\tiny S}
    \bitbox{1}{\tiny S}
    \bitbox{1}{\tiny S}
    \bitbox{1}{\tiny S}
    \bitbox{1}{\tiny S}
    \bitbox{1}{\tiny S}
    \bitbox{1}{\tiny S}
    \bitbox{1}{\tiny S}
    \bitbox{1}{\tiny S}
    \bitbox{1}{\tiny S}
    \bitbox{1}{\tiny S}
    \bitbox{1}{\tiny S}
    \bitbox{1}{\tiny S}
    \bitbox{1}{\tiny S}
    \bitbox{1}{\tiny S}
    \bitbox{1}{\tiny S}
    \bitbox{1}{\tiny S}
    \bitbox{1}{\tiny S}
    \bitbox{1}{\tiny S}
    \bitbox{1}{\tiny S}
    \bitbox{1}{\tiny S}
    \bitbox{1}{\tiny S}
    \bitbox{1}{\tiny S}
    \bitbox{1}{\tiny S}
  \end{rightwordgroup}\\
  \end{bytefield}
}{
  \regItem{Bit 31:0}{GPIOEN}{GPIO Enable. \\
    Contains the enable bit per GPIO line.
  }
}

\regDesc{0x1A10\_1020 - 0x1A10\_103C}{0x0000\_0000}{PADCFG0-7 (Pad Configuration
Registers 0-7)}{
  \begin{bytefield}[rightcurly=.,endianness=big]{32}
  \bitheader{31,30,29,28,27,26,25,24,23,22,21,20,19,18,17,16,15,14,13,12,11,10,9,8,7,6,5,4,3,2,1,0} \\
  \begin{rightwordgroup}{PADCFG0-7}
    \bitbox{1}{\tiny P}
    \bitbox{1}{\tiny P}
    \bitbox{1}{\tiny P}
    \bitbox{1}{\tiny P}
    \bitbox{1}{\tiny P}
    \bitbox{1}{\tiny P}
    \bitbox{1}{\tiny P}
    \bitbox{1}{\tiny P}
    \bitbox{1}{\tiny P}
    \bitbox{1}{\tiny P}
    \bitbox{1}{\tiny P}
    \bitbox{1}{\tiny P}
    \bitbox{1}{\tiny P}
    \bitbox{1}{\tiny P}
    \bitbox{1}{\tiny P}
    \bitbox{1}{\tiny P}
    \bitbox{1}{\tiny P}
    \bitbox{1}{\tiny P}
    \bitbox{1}{\tiny P}
    \bitbox{1}{\tiny P}
    \bitbox{1}{\tiny P}
    \bitbox{1}{\tiny P}
    \bitbox{1}{\tiny P}
    \bitbox{1}{\tiny P}
    \bitbox{1}{\tiny P}
    \bitbox{1}{\tiny P}
    \bitbox{1}{\tiny P}
    \bitbox{1}{\tiny P}
    \bitbox{1}{\tiny P}
    \bitbox{1}{\tiny P}
    \bitbox{1}{\tiny P}
    \bitbox{1}{\tiny P}
  \end{rightwordgroup}\\
  \end{bytefield}
}{
  \regItem{Bit 31:0}{PADCFG0-7}{Pad Configuration Registers. \\
    The pad configuration registers control various aspects of the pads that are
    typically used in ASICs, e.g. drive strength, Schmitt-Triggers, Slew Rate,
    etc. Since those configuration parameters depend on the exact pads used,
    each implementation is free to use the PADCFG0-7 registers in every way it
    wants and also leave them unconnected, if unneeded.
  }
}

Writing to the PADOUTSET address (0x1A10\_1040), the content of the PADOUT register is updated with its content "ored" with the write data.

Writing to the PADOUTCLR address (0x1A10\_1044), the content of the PADOUT register is updated with its content "anded" with the inverted write data.
